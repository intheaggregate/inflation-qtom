\documentclass[paper=a4, fontsize=11pt,twoside]{scrartcl}
\usepackage{authblk}
\usepackage[american]{babel}
\usepackage[babel]{csquotes}
\usepackage[notes, annotation=notes, backend=biber]{biblatex-chicago}
\addbibresource{nabigbib.bib}
\usepackage{setspace}
\doublespacing
\usepackage{outlines}
\usepackage{amsfonts}
\usepackage{hyperref}
\hypersetup{
    colorlinks=true,
    linkcolor=black,
    filecolor=black,      
    urlcolor=blue,
    pdfpagemode=FullScreen,
    citecolor=blue,
    linktoc=all
}
\usepackage{float}
\usepackage[a4paper, pdftex, left=1.7cm,right=1.7cm,top=1.4cm,bottom=2.5cm]{geometry}
\usepackage[a4paper,pdftex]{geometry}
\usepackage[protrusion=true,expansion=true]{microtype}
\usepackage{amsmath,amsfonts,amsthm,amssymb}
\usepackage{graphicx}
\graphicspath{ {./images/} }
\usepackage{wrapfig}
\usepackage{comment}

\makeatletter % Title
\def\printtitle{%
    {\centering \@title\par}}
\makeatother
\makeatletter % Author
\def\printauthor{%
    {\centering \large \@author}}
\makeatother
\makeatother
\newcommand{\HRule}[1]{\rule{\linewidth}{#1}}

\title{\HRule{0.5pt} \\
\LARGE\textbf{\uppercase{Universal Basic Income and Inflation: Reviewing Theory and Evidence}}}
\author{Joshua Z. Miller \\
    June 2021
    \HRule{1.3pt}
    \affil{UBI Center} \\
}

\begin{document}
\begin{titlepage}
\thispagestyle{empty}
\newgeometry{top=1in, left=26mm, right=26mm}
\printtitle
\vspace{2mm}
\printauthor
\date{June 2021}


\begin{figure}[h]
\href{http://www.ubicenter.org}{\includegraphics[width=0.3\textwidth]{ubi_center_logo_wide_blue}}
\centering
\end{figure}
%\begin{comment}
\begin{abstract}
\noindent One of the most common arguments against universal basic income is the idea that it will cause inflation. In this paper I will review the theoretical interactions between an universal basic income and the price level, as well as emerging empirical price effects from cash transfer programs across the world.  I discuss what type of macroeconomic shock universal basic income qualifies as, and I also characterize universal basic income through the lens of orthodox and heterodox monetary theories. Applying Milton Friedman’s Quantity Theory of Money, I theorize about under what conditions universal basic income may cause inflation. Finally, I dissect how public policy choices may affect the inflationary potential of universal basic income. \\
\end{abstract}
%\end{comment}
  \vfill
\tableofcontents
\pagebreak
\restoregeometry
\end{titlepage}
\hypersetup{linkcolor=blue}
\section{Introduction}
A common argument against universal basic income is the idea that it will cause inflation. Whilst there is little polling data to support this point,\footnote{We can look at Google search trends for evidence of this. At the time of writing, the top Google autocomplete result for ``would basic income" was ``would basic income cause inflation". A search for ``would basic income cause inflation" returns almost thirty-five million results.} a cursory glance of the hawkish mentalities many Americans hold about inflation\autocite{10.2307/1991488} explains why this fear would be widely held. An opinion piece\footnote{This article stands out to me in particular because it provides an example that's rooted in a misinterpretation of the theory I will base my model off of. Essentially, the author writes about universal basic income as if it was increasing the money supply. In practice, universal basic income does no such thing, instead merely redistributing money. \autocite{hill:antiubiarticle}.} shows one possible reason for the fears: ``if everyone suddenly had an extra \$10K a year, and everyone knew that everyone had an extra \$10K a year, prices would go up and inflation would rise, thus negating the perceived gains of [universal basic income].'' Their argument is simple: transferred cash goes straight into an increase in the price level. However, these arguments linking universal basic income to inflation ignore the monetary origins of inflation and fail to discern between the interactions the real and nominal worlds that would influence the inflationary potential of universal basic income.

As ``inflation is always and everywhere a monetary phenomenon"\autocite{friedman:counterrev}, the discussion of universal basic income's effects on inflation must be centered on the monetary system. Thus, I will use this paper to survey universal basic income from the perspective of the price system. Little new research is needed; the centuries of research and theoretics on inflation and government spending supply us with a picture of the predictable price effects of universal basic income, and emerging research will supply us with a taste of some empirical evidence to compare our predictions to. 

It is perhaps most simple to present universal basic income as a shock to an aggregate supply-aggregate demand model. However, there a couple issues which make such a model ineffective. One, with no development of the pay-fors (or lack thereof) we cannot seperate the breakdown between the budgetary and ``universal basic income'' shocks, or even determine the existence of a shock. Two, rational expectations literature does not extensively cover the subject of permanent welfare schemes. Therefore, I will not be using an aggregate supply-aggregate demand to examine universal basic income. 

Since universal basic income is often connected to support for more expansionary government economic policy, it may be tempting to utilize Modern Monetary Theory tools to discuss the inflationary effects of universal basic income. However, Modern Monetary Theory literature lacks a formalized treatment of transfer programs in favour of things like a Jobs Guarantee, making universal basic income analysis in that area new territory. 

I have chosen to go with an application of the Quantity Theory of Money, here meaning the idea that the ultimate quantity of money in the economy is the chief determinant of the long-run price level. One reason for this is that Quantity Theory is diverse. The theory precedes classical economics\autocite{history}, and evolved under radically different circumstances than today, thus demonstrating its durability and adaptiveness. Many notable figures such as Copernicus\autocite{copernicus}, Bodin\autocite{bodin}, Locke\autocite{locke}, Mill\autocite{mill}, and -- most famously -- Hume\autocite{hume} contributed to the classical view that greater money per unit of output would translate to increases in prices. Those writers used direct observation of prices and currency flows to establish this, and we can take the example of their writings when discussing inflation in less-developed economies more akin to those of the Enlightenment, as I will seek to do when again discussing microeconometric evidence from Kenya and Mexico. 

Past the classical view, there are many restatements that can be adapted to suit different explanatory purposes, and the idea has been rephrased over time by Pigou\autocite{pigou:value}, Keynes\autocite{keynes:tract}, Fisher\autocite{fisher:ppom}, and Friedman\autocite{friedman:restatement, friedman:quantity}, each with their varying nuances and ease of application to our present discussions. Depending on whether we want to factor in interest rates, different notions of money supply, different measures of real activity, and different indicators of growth, we can use or develop a different restatement of the base identity. 

Quantity Theory also provides a flexible yet intuitive model that we can insert unclear data into, by generalizing terms. Unlike more complicated macroeconomic models, we can use partial data and still reach predictions that can inform debate and policymaking. Additionally, due to the extreme simplicity of the base equation of exchange, we can account for most hard-to-categorize real world phenomena.\footnote{A lot of idiosyncratic shocks can be aggregated as velocity shocks, for example.} 

Furthermore, empirical support behind the Quantity Theory of Money\autocites{candleweb}{10.2307/1805778}{RePEc:fip:fedmwp:89164} gives confidence that our analysis provides practical predictions for the real world. Most heterodox theories of inflation do not provide as strong alignment with the common trend of history. 

\section{A Quantity Theoretic Model of Universal Basic Income} 

My first step in contextualizing universal basic income is establishing what sort of shock universal basic income would be. Take the base equation of exchange, money supply times velocity equals prices times output:

\begin{equation}
MV = PY
\end{equation}

We can turn this equation into an addition of the rates of change of the variables\autocite{friedman:restatement, friedman:quantity}:
\begin{equation}
\frac{1}{M}\frac{\partial M}{\partial t} + \frac{1}{V}\frac{\partial V}{\partial t} = \frac{1}{P}\frac{\partial P}{\partial t} + \frac{1}{Y}\frac{\partial Y}{\partial t}
\label{dQTM}
\end{equation}

In a steady state economy with forward-looking agents, changes to rates of change can effectively model the exogenous shocks that affect the economy, and it is easier to discern exogenous shocks to rates of change than to levels. Thus, when inputting data or a theoretical horizon the rate-of-change model will perform best. Rewriting \ref{dQTM} to isolate price change, we get:

\begin{equation}
\frac{1}{P}\frac{\partial P}{\partial t} = \frac{1}{M}\frac{\partial M}{\partial t} + \frac{1}{V}\frac{\partial V}{\partial t} - \frac{1}{Y}\frac{\partial Y}{\partial t}
\end{equation}

Now, we can discern how universal basic income would cause price change. If universal basic income is deficit or money-financed in the long term, $\frac{1}{M}\frac{\partial M}{\partial t}$ will accelerate, causing inflation \textit{ceteris paribus}. However, if we assume full tax funding in the long term, we can eliminate the M term from 3:

\begin{equation}
\frac{1}{P}\frac{\partial P}{\partial t} = \frac{1}{V}\frac{\partial V}{\partial t} - \frac{1}{Y}\frac{\partial Y}{\partial t}
\end{equation}

Thus, the effects of universal basic income on permanent inflation will solely depend on shifts in velocity and real income. If a universal basic income causes an imminent acceleration in that rate of change of transfers of money, but extant factors inhibit a similar acceleration in real income, universal basic income will be inflationary. If universal basic income does not result in the acceleration of velocity but does effect an increase in real income growth, then universal basic income is actually \textit{deflationary}. Of course, velocity and income acceleration could simply balance out, in which case there would be no price level acceleration. 

Whilst velocity and income are likely to increase because of an universal basic income,\footnote{I discuss this later in the piece.} their growth rates should remain the same. However, switching from means-tested welfare to universal basic income may have beneficial, second-order macroeconomic effects that accelerate growth\autocite{olg1}, thus causing downwards pressure on the price level. In that case, price level growth should decelerate. If an universal basic income is a distortionary force to real income, on the other hand, universal basic income would be inflationary. If an universal basic income has no effect on real income, it should have no price level change effect. 

It is important to note, therefore, the evidence around universal basic income's effect on real income. A paper evaluating the work effects of the Alaska Permanent Fund\autocite{NBERw24312} found no labour supply distortion, which is one of the channels through which a universal basic income could lower real growth. On the other hand, multiple dynamic models constructed around a universal basic income found lower long-run growth rates\autocites{olg1, NBERw27351}.

Note that an universal basic income would probably not cause an increase in velocity growth. Universal basic income would most likely effect a level change. However, I shall still construct a small model to explain velocity changes, and outline one possibility for velocity growth. 

We can define total average velocity as the sum of the transactions of spending, transactions of taxation, and the transactions of saving divided by nominal money demand:

\begin{equation}
V^T = \frac{T^C + T^G + T^S}{M^D}
\end{equation}
From this identity we can gain an appreciation of the adjustments to velocity that an universal basic income may yield. We can hold $M^D$ parsimonious to do this. An universal basic income should be net redistributive, so $T^C$ should increase, as the poorer people in society have a higher consumption ceiling. More tax money would be required, so $T^G$ would increase. We can assume that $T^S$ tends to be low, as savings are held. If universal basic income is net redistributive by way of taxes on savers, who tend to be wealthy, then $T^S$ decreases. It must be noted that an universal basic income funded by way of a value-added tax may indeed decrease $T^C$, dependent on elasticity of consumption. 

Another way we can examine velocity is in the context of the money demand term $M^D$. Since real money demand is endogenous to the real interest rate\autocite{wash1}, velocity will move inversely to the real interest rate. If little real growth is small, the real interest rate will stay the same. If real growth is strong, the real interest rate may increase. 

In the first order, if real growth does not increase, there will not be an increase in both real growth and velocity, and thus no increase in the price level in that it can both offset $V$ shifts or cause them. 

In the long term, if there is strong real growth, velocity growth may either stay the same or increase with the real interest rate. This was the possibility I mentioned earlier, and the only way velocity growth may increase.

Filtering all of these back into the base equation of exchange $MV = PY$, we see that $P$ is ultimately endogenous to $Y$. Since the increase in the $PY$ term is ultimately endogenous to $V$, and that $V$ may or may not be driven by $Y$, $Y$ is the ultimate controlling variable for the level shift. 

We can also consider the findings of Friedman\autocite{10.2307/1832113} and identify that the real interest rate determines the path of the nominal economy. Skipping the distinction of price and real output, we can observe directly that if the real interest rate stays the same, there is no nominal shift and thus no possibility of a price level shift.

Let us now consider the basic real income formula:

\begin{equation}
Y^* = C^* + I^* + G^* + N_x^*
\end{equation}

Now take the basic savings-investment identity:
\begin{equation}
I^* = S^*
\end{equation}

If an universal basic income provides an income effect for a majority of the population, $C^*$ should increase, and $I^*$ should decrease as $S^*$ decreases. Perhaps $N_x^*$ decreases, if citizens purchase more foreign goods with their money, or perhaps it increases if they favor domestic goods. Therefore, the larger the first order economic gains from the net redistribution and cash transfer, the smaller the price level increase. 

It is most likely that housing and consumption goods will constitute the majority of the price level spike.\footnote{This assertion is disputed by a DSGE model constructed to look at the potentials of a city-wide universal basic income  \parencite{city}. That model predicts housing prices fall, if anything, although obviously we cannot say this result is indicative of what can be expected of a national policy, especially as a key component of that model is flight.} Specifically, it would be middle-income housing or consumption goods, as universal basic income would have the effects of compressing income around the center. 

However, it is highly unlikely that the first order growth effect will be particularly strong. The growth, if any, will take time to foment and cannot simply be achieved by shifting money to classes with higher marginal propensities to consume. Thus, universal basic income will probably cause a transitive increase in the price level, regardless of its overall effect on inflation.

\section{Review of Literature \& Evidence}
The most important studies done in the realm of measuring the price effects of universal basic income have been two general equilibrium studies, one looking at transfers from the NGO GiveDirectly in Kenya\autocite{NBERw26600} and the other studying the Mexican food assistance program ``Programa de Apoyo Alimentario".\autocite{NBERw17456}  Both studies evaluated transfer programs to very impoverished places and the latter to a relatively closed economy, so obviously we can expect discrepancies between the evidence and what would happen in reality given a universal basic income in North America. 

Egger et al. find that there was less than 1\% price inflation on average as a result of the transfers. The greatest price effect is to housing, which the authors do not find statistically significant. There is a strong increase in consumption and firm profits, akin to an aggregate demand increase, with spillover effects, and little-to-positive labour hours change. The authors believe they found a transfer multiplier effect of over 2. Additionally, there was a formation of a strong potential growth effect in terms of permanent consumption and income as estimated by the authors. This overall aligns with my model; that is that is, cash transfers caused growth in both permanent output and output level and almost-nonexistent price change. This checks out intuitively: an increase in income resulted in more consumption and more output in an economy that has room to grow, so there will not be additional money-per-unit-of-output that ultimately underlies the Quantity Theory of Money.

The Cunha, Georgi, and Jayachandra study finds again a neglible price inflation effect. However, they do find that in-kind transfers provide deflation compared to cash transfers in isolated villages. Isolated villages likely suffer from imperfect competion and aren't factored in to the monetary partial equilibrium. In the more-developed world with stronger market institutions, we would likely see different. Unfortunately, data for programs like these in the developed world is scarce. 

A new general equilibrium study from The Phillippines\autocite{10.1162/rest_a_01061}, in the vein of the Mexico study, provides us further data to examine. It does initially appear to provide some contradictory evidence towards my findings, but further examinations prove otherwise. The study finds there was marginal price increases in certain foods in villages which received cash due to a demand effect. However, these villages were extremely remote, and these specific foodstuffs were goods with ``relatively high transport and storage costs". Given an opportunity for a more open market clearance, the demand effect may be subsided by a competition effect as shown elsewhere. 

Some further evidence we can examine\autocite{santens} is Alaska's inflation statistics. Alaska has had a universal basic income via the Alaska Permanent Fund Dividend, which has provided each resident between \$500 and \$3,000 per year funded by state oil assets, since 1982. Since implementing that program, Alaska has had lower inflation than the rest of the United States\autocite{fred1}, whereas before it had equal. While no formal causal study has yet confirmed whether this supports the potential ``deflation" effect of universal basic income I discussed earlier, it does provide suggestive evidence that it is possible and that a US-based universal basic income would not cause significant inflation. We can also appreciate the post-transfer deflation from Kuwait\autocite{imf1}, where Kuwait granted a \$3,500 transfer to all its citizens in 2011\autocite{imfkuw}.

\section{Conclusion}
Universal basic income will cause inflation depending on whether or not it spurs strong real growth. Depending on the shape of the benefits and tax scheme, different growth effects will occur, so each scenario is unique and must be treated as such. When studying inflation and universal basic income, the monetary nature of inflation must be paid heed, and so must be the potential for deflation. A suitable policy regime will need to be aware of the particular influence the cash transfer program has on the equilibrium path of nominal output. 

A new method should be developed to evaluate cash transfer programs and other welfare programs within a short-to-medium-run macroeconomic horizon. The central bank may indeed have the final say in terms of the price level change, and different outcomes may occur depending on the path the central bank sets for nominal income. To this point, further study should be devoted to cash transfers and general equilibrium in the developed world. 
\pagebreak
\printbibliography
\end{document}
